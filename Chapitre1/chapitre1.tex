\setcounter{mtc}{5} %indique le numero reel du chapitre, pour la mini table des matieres
\chapter{Project Scope}
\minitoc  %insert la minitoc

\graphicspath{{Chapitre1/figures/}}
%==============================================================================
\pagestyle{fancy}
\fancyhf{}
\fancyhead[R]{\bfseries\rightmark}
\fancyfoot[R]{\thepage}
\renewcommand{\headrulewidth}{0.5pt}
\renewcommand{\footrulewidth}{0pt}
\renewcommand{\chaptermark}[1]{\markboth{\MakeUppercase{\chaptername~\thechapter. #1 }}{}}
\renewcommand{\sectionmark}[1]{\markright{\thechapter.\thesection~ #1}}

\begin{spacing}{1.2}
%==============================================================================

\section*{Introduction}
\section{Host company presentation} 
\subsection{Presentation of Kaoun}
\subsection{Presentation of Flouci}
\section{Project overview}
\subsection{Project Context}
\subsection{Study of the existing}
\subsection{Project goals}
\section{Methodology}
\subsection{Agile methodology}
\subsection{SCRUM methodology}
\subsection{Lean software development}
\section*{Conclusion}


Une etude theorique \cite{YOUSFI2015} peut contenir l'une et/ou l'autre de ces deux parties :
Elle est en general realisee quand on va developper un module supplementaire sur un 
logiciel existant, ou si on va modifier une application existante. L'etude de l'existant
consiste e expliquer ce qui existe deje dans votre environnement de travail.


La conclusion est en general sans numerotation, et n'apparaet pas dans la table des matieres.

C'est une etude assez detaillee sur ce qui existe sur le marche ou dans la litterature (d'oe 
le terme etat de l'art), qui permet de repondre e la problematique. L'idee ici est de faire 
un comparatif entre les solutions existantes, mais surtout d'analyser le resultat de cette 
comparaison et de dire pourquoi ne sont-elles pas satisfaisantes pour repondre e votre 
problematique.
\begin{figure}[!ht]\centering
\includegraphics[scale=0.9]{art.jpg}
\caption{state of the art}
\label{fig:fig1}
\end{figure}





%==============================================================================
\end{spacing}
