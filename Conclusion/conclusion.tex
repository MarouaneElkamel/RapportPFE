\chapter{General Conclusion and Perspectives }
%==============================================================================
\pagestyle{fancy}
\fancyhf{}
\fancyhead[R]{\bfseries\rightmark}
\fancyfoot[R]{\thepage}
\renewcommand{\headrulewidth}{0.5pt}
\renewcommand{\footrulewidth}{0pt}
\renewcommand{\chaptermark}[1]{\markboth{\MakeUppercase{\chaptername~\thechapter. #1 }}{}}
\renewcommand{\sectionmark}[1]{\markright{\thechapter.\thesection~ #1}}

\begin{spacing}{1.2}
%==============================================================================


\hspace{\parindent} Over the course of four months, we have designed and implemented our  project despite its highly challenging nature. We have expanded the Flouci ecosystem and added the ability to perform online integrations for any developer.
\newline

The project has involved many pieces: a platform that allows developers to manage their integrations, a module that can be integrated into any e-commerce website, the existing APIs of the Flouci ecosystem (including both the payment API and the wallets API).   
\newline

We started our work by setting up a modern software development discipline. We followed the Kanban methodology and created our project board from the first day. We defined a DevOps pipeline that involves different steps, from testing to packaging and deploying. Finally, we followed a strict TDD to ensure the best possible testing coverage. 
\newline

We began to implement the core functionalities only after ensuring that we are able to write the highest quality possible code using the latest development standards. 
\newline

First, we started implementing the developer's platform. A high level of flexibility and adaptiveness was required to complete this project in a sustainable manner: during the course of the project, the company decided to build all of its future projects in React. This decision prompted us to entirely learn React to complete the platform and ensure it would be compatible with all future company products. We challenged ourselves to build the front end in React.
\newline

After finishing our platform, we dived into the next sprint and started developing the integration module. The most difficult part of the project was to make the module as portable as possible, and required that we implemented it in pure JavaScript.  
\newline

In the end, we delivered a platform to create integration apps, as well as a simple module that could use those apps to add the Flouci payment method on any website. A developer only needs to know how to perform a simple REST call to complete integration with the Flouci payments platform.
\newline

At this point, our project could continue to evolve and add more functionalities:
\begin{itemize}
	\item \textbf{Platform:} We could add more customization to our integration app and allow developers to add their users and items. After that, we can add more metrics relevant to items or users, such as most sold items or most active customers.
	\item \textbf{Checkout module:} We could add the possibility to pay with Flouci login credentials, or create a button that redirects to the app on mobile devices to perform payments without having to scan any QR codes.
\end{itemize} 

%==============================================================================
\end{spacing}