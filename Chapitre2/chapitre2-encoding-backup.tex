
\setcounter{chapter}{1}
\chapter{Conception}
\minitoc %insert la minitoc
\graphicspath{{Chapitre2/figures/}}

%\DoPToC

%==============================================================================
\pagestyle{fancy}
\fancyhf{}
\fancyhead[R]{\bfseries\rightmark}
\fancyfoot[R]{\thepage}
\renewcommand{\headrulewidth}{0.5pt}
\renewcommand{\footrulewidth}{0pt}
\renewcommand{\chaptermark}[1]{\markboth{\MakeUppercase{\chaptername~\thechapter. #1 }}{}}
\renewcommand{\sectionmark}[1]{\markright{\thechapter.\thesection~ #1}}

\begin{spacing}{1.2}
%==============================================================================
\section*{Introduction}
La partie conception de l'application \textbf{\textsl{n'est pas toujours obligatoire}}. En effet, quand notre travail consiste en une etude theorique, ou une mise en place d'un systeme par exemple,
il est inutile voire obsolete de faire un diagramme de classes ou de sequence.\\
Quand il s'agit de developpement, par contre, la partie conception s'impose.
\section{Recommadations}
En general,
il faut suivre les regles suivantes :
\begin{itemize}
	\item Choisir une methodologie de travail : un processus unifie, une methode agile;
\end{itemize}
\section{Diagrammes}
Il faut Bien choisir les diagrammes adequats pour votre application. En general, les
diagrammes obligatoires sont les diagrammes de cas d'utilisation, de classe et de
sequence. Vous pouvez ajouter en plus le diagramme qui vous semble pertinent :
par exemple, pour une application sur plusieurs tiers, il est interessant de
montrer le diagramme de deploiement;
\begin{itemize}

\item Les diagrammes doivent etre clairs, lisibles et bien expliques, sans pour autant
nous submerger de details. Des explications trop longues deviennent ennuyeuses;
\item Si un diagramme est trop grand, vous pouvez le diviser, le representer sous
forme de plusieurs diagrammes, ou vous abstraire de certains details. Si c'est
impossible, imprimez le sur une grande page (A3), quitte e la plier ensuite. Le
plus important est que tous les mots soient lisibles.

\end{itemize}
\subsection{Diagramme de Sequence}
Un diagramme de sequence :
\begin{itemize}
	\item Represente un scenario possible qui se deroule dans un cas d'utilisation.
	Vous n'etes donc pas obliges de montrer tous les cas d'execution possibles;
	\item Represente l'interaction entre les objets : donc normalement, toutes les
	instances definies dans un diagramme de sequences doivent correspondre
	 e des classes qui se trouvent dans le diagramme des classes;
	 \item Il existe parfois des dizaines de diagrammes de sequences possibles. Choisissez certains d'entre eux e mettre dans le rapport (2 ou 3). Privilegiez les diagrammes les plus importants (et non, l'authentification n'en fait pas partie!).
\end{itemize}
\subsection{Diagramme de Classes}
Un diagramme de classes :
\begin{itemize}
\item Doit etre fidele e l'architecture logicielle choisie. Si vous utilisez le MVC,
alors les trois couches doivent etre representees dans le diagramme de classes grece aux packages;
\item Les stereotypes sont fortement conseilles. Si vous developpez une
application web, n'hesitez pas e utiliser les stereotypes de la figure \ref{fig:fig2} :
\begin{figure}[!ht]\centering
\includegraphics[scale=0.9]{stereotypes.jpg}
\caption{Les stereotypes}
\label{fig:fig2}
\end{figure}
\item Attention e ne pas confondre classes et tables : evitez la tentation de
mettre des id partout.

\end{itemize}

\section*{Conclusion}
Faire ici une petite recapitulation du chapitre, ainsi qu'une introduction du chapitre suivant.





%==============================================================================
\end{spacing}
