\chapter*{General Introduction}
\graphicspath{{Introduction/figures/}}
\addcontentsline{toc}{chapter}{General Introduction}
\begin{spacing}{1.1}
%==================================================================================================%

The word "fintech" refers to dynamic, innovative companies that apply new technologies to compete with or supplement traditional financial services. Their purpose is to offer simple, efficient and cost-effective financial services to the general public and professionals alike. In short, they are companies, usually startups, combining finance and technology to create entirely new lines of products that are able to streamline, improve or democratize traditional bank activities.\newline

Kaoun, a Tunisian fintech startup, is developing it's first product, Flouci. Flouci is a new mobile payment solution that will allow users to open bank accounts remotely using electronic KYC (Know Your Consumer), send peer to peer transactions, pay merchants online and in person, and eventually create digital financial histories that will allow for alternative credit scoring.   \newline


As the Tunisian market for goods and services is moving towards online payments, Kaoun decided to create its own developer's checkout API via Flouci. This project will expand Flouci to the online payments world and unleash the full potential of the product, while also expanding the potential for Flouci to move into new markets and opening doors for unlimited integrations.\newline


The following report is a synthesis of the efforts done to build the Flouci developer's API.
To detail the process of our work, we have divided this report into five chapters representing the different aspects of our project.

In the first chapter entitled 'Project Scope', we started with a presentation of the host company. Afterward,  we gave an overview of our project and we detailed the followed methodology for its realization.

In the second chapter, entitled 'Requirements Analysis and Specification', we studied the existing solutions available in the market and went through a detailed study of functional and non functional requirements of our project.

In the third chapter, entitled 'Sprint 1:  Project Launch', we presented the development disciplines and rules we set during our project development life cycle. This is a detailed explanation of the development practices needed to start our project development in the most efficient way possible. 

In the fourth chapter, entitled 'Sprint 2:  User and App Management', we started the implementation of our platform and tackled the two main components: the user and the app.

In the fifth chapter, entitled 'Sprint 3:  Checkout Module', we made a module that could integrate the Flouci app into any website as a payment method.
 
We close our work with a general conclusion in which we evaluate our contribution, as well as develop our vision for the project's potential improvements.



\end{spacing}
