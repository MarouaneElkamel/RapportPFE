\chapter*{General Introduction}
\graphicspath{{Introduction/figures/}}
\addcontentsline{toc}{chapter}{General Introduction}
\begin{spacing}{1.2}
%==================================================================================================%

De nos jours, les consommateurs du secteur agroalimentaire sont devenus de plus en plus exigeants ce qui a créé chez les industriels, un besoin de maîtriser et d'optimiser la qualité des aliments, rendant ainsi, la sécurité des denrées alimentaires et la traçabilité, des atouts stratégiques pour la filière agroalimentaire. En effet, la pression exercée par les clients quant à la qualité du produit de consommation, conduit les entreprises agroalimentaires et les autorités à contrôler non seulement les produits finis mais toute la chaîne de production. Or, + fabriquer un produit sain, salubre et de qualité implique de répondre aux exigences réglementaires relatives à la qualité et à la sécurité des denrées alimentaire\newline


Ainsi, pour maintenir la compétitivité industrielle, toute entreprise du secteur, doit passer par le renforcement de ses capacités à fournir à ses clients des produits conformes aux exigences législatives et réglementaires en vigueur.\newline


Face à une demande de plus en plus importante des clients à la communauté agro-alimentaire pour qu’elle démontre son aptitude à identifier et à maîtriser les dangers liés à la sécurité des denrées alimentaires;cette dernière a multiplié les initiatives pour établir des règles plus ou moins volontaires. Au sein d’une révolution des référentiels de qualité, la norme FSSC 22000  est le dernier standard de certification pour les fabricants de produits alimentaires. Sa nouvelle version a été publiée le 3 juin 2019. Cette version 5 du référentiel de Food Safety de référence est une version révisée de FSSC 22000 v 4.1, officiellement lancée en juillet 2017.\newline


Consciente des avantages apportés par cette norme en termes de la gestion de la qualité des produits mis à la disposition de ses clients, l’entreprise  Société Tunisienne des Industries Alimentaire Laitières (STIAL) Délice Danone s’oriente vers une politique d’amélioration continue et pour renforcer plus son système de management de la sécurité des aliments , la société a introduit la norme FSSC 22000, qui représente l’une des approches les plus exhaustives de ce système et combine les avantages d´un outil de management commercial lié à sécurité des denrées alimentaires et les processus d'affaires avec la capacité de répondre aux exigences de la clientèle mondiale.\newline


Les modifications  apportées au référentiel FSSC 22 000 en sa dernière version 5 va  rendre obligatoire de mettre  à jour le système de sécurité des denrées alimentaire au niveau de la STIAL Délice Danone puisque les audits selon les exigences de la version 4.1 ne sont autorisés que jusqu'au 31 décembre 2019 et tous les certificats FSSC 22000 v 4.1 délivrés deviendront invalides après le 29 juin 2021. Les audits de mise à niveau par rapport aux exigences du système FSSC 22000 v 5 seront à effectuer entre le 1er janvier 2020 et le 31 décembre 2020. Dans ce sens, l'objectif assigné à ce stage de fin d’études se résume dans la participation à la migration  de ce système selon le protocole de certification FSSC 22 000 version 5 et plus particulièrement à la mise en place des processus de l’entreprise exigée par la nouvelle version de l’ISO 22 000 : 2018 , un des constituants de la FSSC 22000 version 5.\newline


Pour cela, un premier chapitre présente le système de management de la sécurité alimentaire selon le protocole de certification FSSC 22 000 V5 et ses principales exigences , Un deuxième et un troisième chapitre s'intéressent successivement  à l’identification, la mise en place des processus de la société  et à l’analyses des risques relative à chaque processus ainsi que les actions à mettre en oeuvre face aux risques et opportunités selon la norme ISO 22000 : 2018\newline



\end{spacing}
