\chapter*{Introduction Generale}

\addcontentsline{toc}{chapter}{Introduction Generale}
\begin{spacing}{1.2}
%==================================================================================================%

Pour ecrire un bon rapport \cite{SFAXI2015} de projet en informatique, il existe certaines regles e respecter. Certes, chacun ecrit son rapport avec sa propre plume et sa propre signature, mais certaines regles restent universelles    \cite{Latex}.\\

\textbf{La Table de matiere} est la premiere chose qu'un rapporteur va lire. Il faut qu'elle soit :
\begin{itemize}
\item Assez detaillee \footnote{Sans l'etre trop}. En general, 3 niveaux de numeros suffisent;
\item Votre rapport doit etre reparti en chapitres equilibres, e part l'introduction et la conclusion, naturellement plus courts que les autres;
\item Vos titres doivent etre suffisamment personnalises pour donner une idee sur votre travail. eviter le : e Conception e,  mais privilegier : e Conception de l'application de gestion des $...$ e Meme s'ils vous paraissent longs, c'est mieux que 
d'avoir un sommaire impersonnel. \\
\end{itemize}

\textbf{Une introduction} doit etre redigee sous forme de paragraphes bien ficeles. Elle est
normalement constituee de 4 grandes parties :
\begin{enumerate}
\item Le contexte de votre application : le domaine en general, par exemple le domaine du web, de BI, des logiciels de gestion ?
\item La problematique : quels sont les besoins qui, dans ce contexte le, necessitent la realisation de votre projet?
\item La contribution : expliquer assez brievement en quoi consiste votre application, sans entrer dans les details de realisation. Ne pas oublier qu'une introduction est
 censee introduire le travail, pas le resumer; 
 \item La composition du rapport : les differents chapitres et leur composition. Il n'est pas necessaire de numeroter ces parties, mais les mettre plutet sous forme de paragraphes successifs bien lies.
\end{enumerate}






\end{spacing}


