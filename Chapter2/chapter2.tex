
\setcounter{chapter}{1}
\chapter{ Présentation du protocole de Certification FSSC 22 000 version 5}
\minitoc %insert la minitoc
\graphicspath{{Chapter2/figures/}}

%\DoPToC

%==============================================================================
\pagestyle{fancy}
\fancyhf{}
\fancyhead[R]{\bfseries\rightmark}
\fancyfoot[R]{\thepage}
\renewcommand{\headrulewidth}{0.5pt}
\renewcommand{\footrulewidth}{0pt}
\renewcommand{\chaptermark}[1]{\markboth{\MakeUppercase{\chaptername~\thechapter. #1 }}{}}
\renewcommand{\sectionmark}[1]{\markright{\thechapter.\thesection~ #1}}

\begin{spacing}{1.2}
%==============================================================================
\section*{Introduction}
Dans ce chapitre on va introduire avec un bref historique concernant les  normes de sécurité alimentaire pour ensuite présenter le protocole de certification FSSC 22 000 version 5, ses constituants ainsi que les principaux changements de la dernière version de la norme ISO 22000

\section{Les normes de sécurité alimentaire}
\section{FSSC 22000: Food Safety System Certification 22000}
\subsection*{Introduction}
\subsection{Avantages de FSSC 22000}
\subsection{Constituants de FSSC 22000}
\subsubsection{La norme ISO 22000}
\paragraph{Naissance de l’ISO 22000}


%==============================================================================
\end{spacing}
