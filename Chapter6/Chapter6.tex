
\setcounter{chapter}{5}
\chapter{Sprint 4: Payment Integration}
\minitoc %insert la minitoc
\graphicspath{{Chapter6/figures/}}

%\DoPToC
%==============================================================================
\pagestyle{fancy}
\fancyhf{}
\fancyhead[R]{\bfseries\rightmark}
\fancyfoot[R]{\thepage}
\renewcommand{\headrulewidth}{0.5pt}
\renewcommand{\footrulewidth}{0pt}
\renewcommand{\chaptermark}[1]{\markboth{\MakeUppercase{\chaptername~\thechapter. #1 }}{}}
\renewcommand{\sectionmark}[1]{\markright{\thechapter.\thesection~ #1}}

\begin{spacing}{1.2}

%==============================================================================
\section*{Introduction}
Ce chapitre porte sur la partie pratique ainsi que la bibliographie.

\section{Outils et langages utilises}
L'etude technique peut se trouver dans cette partie, comme elle peut etre faite en
parallele avec l'etude theorique (comme le suggere le modele 2TUP).
Dans cette partie, il faut essayer de convaincre le lecteur de vos choix en termes de
technologie. Un etat de l'art est souhaite ici, avec un comparatif, une synthese et un choix 
d'outils, meme tres brefs.
\section{Presentation de l'application}
Il est tout e fait normal que tout le monde attende cette partie pour coller e souhait toutes les images
correspondant aux interfaces diverses de l'application si chere e votre coeur, mais
abstenez vous! Il FAUT mettre des imprime ecrans, mais bien choisis, et surtout, il faut les scenariser : Choisissez un scenario d'execution, par exemple la creation d'un 
nouveau client, et montrer les differentes interfaces necessaires pour le faire, en
expliquant brievement le comportement de l'application. Pas trop d'images, ni trop de
commentaires : concis, encore et toujours.

evitez ici de coller du code : personne n'a envie de voir le contenu de vos classes.
Mais  vous  pouvez inserer des snippets (bouts de code) pour montrer certaines
fonctionnalites \cite{ELKALMEL2019}\cite{Latex}, si vous en avez vraiment besoin. Si vous voulez montrer une partie de votre code, les etapes d'installation ou de configuration, vous pourrez les mettre dans l'annexe.
\subsection{Exemple de tableau}

Vous pouvez utiliser une description tabulaire d'une eventuelle comparaison entre les travaux existants. Ceci est un exemple de tableau: Tab \ref{tab:exple}.

\begin{table}[ht]
	\centering
	\caption{Tableau comparatif}
	\footnotesize
	\begin{tabularx}{\linewidth}{|>{\bfseries \vspace*{\fill}}X ||>{\centering{}\vspace*{\fill}}X|>{\centering{}\vspace*{\fill}}X|>{\centering{}\vspace*{\fill}}X|>{\vspace*{\fill}}X<{\centering{}}|}	
			\hline 
			& \bfseries Col1 & \bfseries Col2 &\bfseries Col3 &\bfseries Col4\\
			\hline \hline
			Row1		&		&	X	&		&		\\
			Row2		&	X	&		&		&		\\
			Row3		&	X	&	X	&	X	&	X	\\
			Row4		&	X	&		&	X	&	X	\\
			Row5		&	X	&		&	X	&	X	\\
			Row6		&	X	&		&	X	&	X	\\
			Row7		&	X	&		&	X	&		\\
			Row8		&	X	&	X	&	X	&		\\
			\hline
	\end{tabularx}
	\label{tab:exple}
\end{table}

\subsection{Exemple de Code}
Voici un exemple de code Java, avec coloration syntaxique \ref{code:java}.

\begin{lstlisting}[rulecolor=\color{white}]
\end{lstlisting}

\begin{lstlisting}[label=code:java,caption=Helloworld Java,language=java]
	public class HelloWorld {
	//la methode main
    public static void main(String[] args) {
        System.out.println("Hello, World");
    }

}
\end{lstlisting}

\section{Remarques sur la bibliographie}
Votre bibliographie doit repondre e certains criteres, sinon, on vous fera encore et
toujours la remarque dessus (et parfois, meme si vous pensez avoir tout fait comme il
 faut, on peut vous faire la remarque quand meme : chacun a une conception tres
personnelle de comment une bibliographie devrait etre).\\
\begin{itemize}
\item Une bibliographie dans un bon rapport doit contenir plus de livres et d'articles 
que de sites web : apres tout c'est une biblio. Privilegiez donc les ouvrages
reconnus et publies pour vos definitions, au lieu de sauter directement sur le premier article wikipedia;
 \item Les elements d'une bibliographie sont de preference classes par ordre
alphabetique, ou par themes (et ordre alphabetique pour chaque theme);
\item Une entree bibliographique doit etre sous la forme suivante :
\begin{itemize}
\item Elle doit contenir un identifiant unique: represente soit par un numero
[1] ou par le nom du premier auteur, suivi de l'annee d'edition [Kuntz, 1987];
\item Si c'est un livre : Les noms des auteurs, suivi du titre du livre, de l'editeur, 
ISBN/ISSN, et la date d'edition;
\item Si c'est un article : Les noms des auteurs, le titre , le journal ou la
conference, et la date de publication;
\item Si c'est un site web ou un document electronique : Le titre, le lien et la date 
de consultation;
\item Si c'est une these : nom et prenom, titre de la these, universite de
soutenance, annee de soutenance, nombre de pages;
\item Exemples : 
\begin{description}
\item $[Bazin, 1992]$ BAZIN R., REGNIER B. Les traitements antiviraux et leurs essais
therapeutiques. Rev. Prat., 1992, 42, 2, p.148-153.\\
\item $[Anderson,1998]$ ANDERSON P.JF. Checklist of criteria used for evaluation of metasites.
[en ligne]. Universite du Michigan, Etats Unis. Site disponible sur :\\
http://www.lib.umich.edu/megasite/critlist.html.(Page consultee le 11/09/1998).
\end{description}
\item Dans le texte du rapport, on doit obligatoirement citer la reference en  faisant appel e son identifiant, juste apres avoir utilise la citation. Si ceci n'est pas fait dans les regles, on peut etre accuse de plagiat.
\end{itemize} 
\end{itemize} 

\section*{Conclusion}
Voile.

%==============================================================================
\end{spacing}
