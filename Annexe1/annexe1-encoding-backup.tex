\setcounter{figure}{0} 
\setcounter{table}{0}
\setcounter{footnote}{0}
\setcounter{equation}{0}
\pagestyle{fancy}
\fancyhf{}
\renewcommand{\chaptermark}[1]{\markboth{\MakeUppercase{#1 }}{}}
\renewcommand{\sectionmark}[1]{\markright{\thesection~ #1}}
\fancyhead[RO]{\bfseries\rightmark}
\fancyhead[LE]{\bfseries\leftmark}
\fancyfoot[RO]{\thepage}
\fancyfoot[LE]{\thepage}
\renewcommand{\headrulewidth}{0.5pt}
\renewcommand{\footrulewidth}{0pt}

\makeatletter
\renewcommand\thefigure{A.\arabic{figure}}
\renewcommand\thetable{A.\arabic{table}} 
\makeatother

\chapter{Annexe : Remarques Diverses}
\graphicspath{{Annexe1/figures/}}
%==========================================================================

%    Annexe

%===========================================================================
\begin{itemize}
\item Un rapport doit toujours etre bien numerote;
\item De preference, ne pas utiliser plus que deux couleurs, ni un caractere fantaisiste; 
\item Essayer de toujours garder votre rapport sobre et professionnel; 
\item Ne jamais utiliser de je ni de on, mais toujours le nous (meme si tu as tout fait tout seul); 
\item Si on n'a pas de paragraphe 1.2, ne pas mettre de 1.1;
\item TOUJOURS, TOUJOURS faire relire votre rapport e quelqu'un d'autre (de preference qui n'est pas du domaine) pour vous corriger les fautes d'orthographe et de franeais;
\item Toujours valoriser votre travail : votre contribution doit etre bien claire et mise en evidence; 
\item Dans chaque chapitre, on doit trouver une introduction et une conclusion;
\item Ayez toujours un fil conducteur dans votre rapport. Il faut que le lecteur suive un raisonnement bien clair, et trouve la relation entre les differentes parties;
\item Il faut toujours que les abreviations soient definies au moins la premiere fois oe elles sont utilisees. Si vous en avez beaucoup, utilisez un glossaire.
\item Vous avez tendance, en decrivant  l'environnement materiel, e parler de votre ordinateur, sur lequel vous avez developpe : ceci est inutile. Dans cette partie, on ne cite que le materiel qui a une influence sur votre application. Que vous l'ayez developpe sur Windows Vista ou sur Ubuntu n'a aucune importance;
\item Ne jamais mettre de titres en fin de page; 
\item Essayer toujours d'utiliser des termes franeais, et eviter l'anglicisme. Si certains termes  sont plus connus en  anglais, donner leur equivalent en franeais la premiere fois que vous les utilisez, puis utilisez le mot anglais, mais en italique;
\item eviter les phrases trop longues : clair et concis, c'est la regle generale !\\

\newpage

\textbf{Rappelez vous que votre rapport est le visage de votre travail : un mauvais rapport peut eclipser de l'excellent travail. Alors pretez-y l'attention necessaire.}

 
\begin{figure}[!ht]\centering
\includegraphics[scale=0.5]{ingenieur.jpg}
\end{figure}
\end{itemize}

